%!TEX TS-program = pdflatexmk
%!TEX root = doc.tex

% Copyright (c) 2018 - 2021, Martin Scheidt (ISC license)
% Permission to use, copy, modify, and/or distribute this file for any purpose with or without fee is hereby granted, provided that the above copyright notice and this permission notice appear in all copies.

\begin{versionhistory}
  %\vhEntry{<Version>}{<Date>}{<Author1>|<Author2>|...}{<Changes>}
  \vhEntry{0.1}{2018-09-14}{MS}{
    Basic concept of a library with railway topology symbols and some examples.
  }
  \vhEntry{0.2}{2018-12-19}{MS}{
    Added transmitters and minor improvements.
  }
  \vhEntry{0.3}{2019-04-04}{MS}{
    Moved snippet folder to root folder and defined and used color foreground and background.
  }
  \vhEntry{0.4}{2019-07-21}{MS}{
    Reworked library for common tikz library layout.
  }
  \vhEntry{0.5}{2020-01-14}{MS}{
    Introducing new syntax and providing a documentation.
  }
  \vhEntry{0.5.1}{2020-02-10}{MS}{
    Modified symbol "end of movement authority"; added symbols "braking point" and "danger point".
  }
  \vhEntry{0.6}{2021-01-02}{MS}{
    Added symbols for "direction control", "track marking", "pylons" and electric wiring; changed symbol for "friction bufferstop"; created an encapsulating package for future flexibilty - changed load command for library to \textbackslash usepackage\{tikz-trackschematic\}.
  }
\end{versionhistory}
